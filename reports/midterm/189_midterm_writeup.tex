\documentclass[final]{article}

% if you need to pass options to natbib, use, e.g.:
% \PassOptionsToPackage{numbers, compress}{natbib}
% before loading nips_2017
%
% to avoid loading the natbib package, add option nonatbib:
% \usepackage[nonatbib]{nips_2017}

\usepackage{nips_2017}

% to compile a camera-ready version, add the [final] option, e.g.:
% \usepackage[final]{nips_2017}

\usepackage[utf8]{inputenc} % allow utf-8 input
\usepackage[T1]{fontenc}    % use 8-bit T1 fonts
\usepackage{hyperref}       % hyperlinks
\usepackage{url}            % simple URL typesetting
\usepackage{booktabs}       % professional-quality tables
\usepackage{amsfonts}       % blackboard math symbols
\usepackage{nicefrac}       % compact symbols for 1/2, etc.
\usepackage{microtype}      % microtypography

% \usepackage{lmodern}

\title{Formatting instructions for NIPS 2017}

% The \author macro works with any number of authors. There are two
% commands used to separate the names and addresses of multiple
% authors: \And and \AND.
%
% Using \And between authors leaves it to LaTeX to determine where to
% break the lines. Using \AND forces a line break at that point. So,
% if LaTeX puts 3 of 4 authors names on the first line, and the last
% on the second line, try using \AND instead of \And before the third
% author name.

\author{
  Forest Kobayashi \\
  Department of Mathematics\\
  Harvey Mudd College\\
  Claremont, CA 91711 \\
  \texttt{fkobayash@hmc.edu} \\
  %% examples of more authors
  \And
  Jacky Lee \\
  Department of Mathematics\\
  Department of Computer Science \\
  Harvey Mudd College \\
  Claremont, CA 91711 \\
  \texttt{jlee@hmc.edu}
}

\begin{document}
% \nipsfinalcopy is no longer used

\maketitle

\begin{abstract}
  Stock data is very difficult to analyze using classical methods, in
  large part due to heavy involvement of humans in stock pricing. In
  this paper, we examine a new approach to analyzing time-series stock
  data, particularly in the context of grouping correlated stocks.
\end{abstract}

\section{Introduction}
Here is a rough table of contents for our paper, so that the reader
has a rough idea of where we'll be going:
\begin{enumerate}
\item Motivation (challenges to modelling stock data)
\item Long-term goal for the model / ideal model architecture
\item Classification
\item Findings
\item Conclusion
\end{enumerate}

\section{}

\section{Goal for the Model}


\end{document}
